\documentclass[11pt,a4paper]{article}
\usepackage{graphicx}
\usepackage{geometry}
%\geometry{legalpaper, landscape, margin=2in} %to convert into slides
\usepackage[italian]{babel}
\usepackage{pdfpages}
\usepackage{amsmath}
\usepackage{amsfonts}
\usepackage{bm} %Package for bold greek letters. Use for example $\bm{\Chi}$
\usepackage{eucal}
\usepackage{hyperref}

\begin{document}

\title{
}
\author{\bf{Two particles decay useful formulae}}
\date{}
\maketitle

\noindent
Suppose we have the decay process $B \rightarrow K + \pi$ with given masses.
\newline
If, in the lab, we have the following four-momenta:
\begin{equation*}
    p_\pi^\mu = (E_\pi, \vec{p}_\pi) \qquad p_K^\mu = (E_K, \vec{p}_K)
\end{equation*}
\noindent
then the invariant mass is:
\begin{equation*}
    s = \eta_{\mu\nu} \hspace{0.5mm} p_{tot}^\mu \hspace{0.5mm} p_{tot}^\nu = m_\pi^2 + m_K^2 + 2\left( \sqrt{m_\pi^2 + p_\pi^2}\hspace{0.5mm}\sqrt{m_K^2 + p_K^2} - p_\pi \hspace{0.5mm} p_K\cos(\theta) \right)
\end{equation*}
\noindent
If we apply a gaussian distribution (e.g., due to the resolution of the detector) to $p_\pi$ and another one to $p_K$ we have to consider these changes
\begin{equation*}
    p_\pi \rightarrow p_\pi + \delta p_\pi \qquad p_K \rightarrow p_K + \delta p_K 
\end{equation*}
First, the two factor squared under the square roots: they bring only positive contributions to invariant mass of the system. Second, the other two factors multiplying $\cos(\theta)$: they bring both positive and negative factors.
At the end, the mean of invariant mass under a sample of $10^4$ events is switched to higher mass than $m_B$.
For this reason this effect is called {\bf "smearing"}.
\end{document}